% Chapter 1

\chapter{项目任务} % Main chapter title

\label{Chapter1} % For referencing the chapter elsewhere, use \ref{Chapter1} 

%----------------------------------------------------------------------------------------

% Define some commands to keep the formatting separated from the content 
\newcommand{\keyword}[1]{\textbf{#1}}
\newcommand{\tabhead}[1]{\textbf{#1}}
\newcommand{\code}[1]{\texttt{#1}}
\newcommand{\file}[1]{\texttt{\bfseries#1}}
\newcommand{\option}[1]{\texttt{\itshape#1}}

本项目是清华大学计算机科学与技术系开设的《数据库系统概论》课程项目。

项目的任务是实现一个单用户的关系数据库管理系统。该项目分为四个功能模块:
\begin{enumerate}[(1)]
\item 记录管理模块:该模块是DBMS的文件系统,管理存储数据库记录以及元数据的文件。该模块依赖于我们预先给定的一个页式文件I/O系统,在此基础上扩展而成。
\item 索引模块:为存储在文件中的记录建立B+树索引,加快查找速度。
\item 系统管理模块:实现基本的数据定义语言(DDL),实现解析器来解析命令行。
\item 查询解析模块:解析SQL语句,能将输入的SQL语句解析成关系代数表达式,并生成查询执行计划,访问文件系统执行查询,输出查询结果。
\end{enumerate}

在上述功能的基础上, 还可以对该系统进行个性化的功能扩展及性能优化,内容包括但不限于:
\begin{enumerate}[(1)] 
\item 查询优化:基于对查询计划代价的估计,为给定查询选择最有效的查询执行计划。
\item 支持属性域约束和外键约束。
\item 创建数据表时支持更多的数据类型。例如decimal, date等。
\item 支持三个或以上表的连接。
\item 支持更多SQL语句。例如聚集查询AVG,SUM,MIN,MAX,GROUP BY等。
\item 支持模糊查询。例如LIKE关键字以及“$\%$, *, ?”等通配符。
\item 提供类似MySQL front的图形化UI。
\end{enumerate}


%----------------------------------------------------------------------------------------

