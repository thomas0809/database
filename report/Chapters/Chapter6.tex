% Chapter Template

\chapter{实验总结} % Main chapter title

\label{Chapter6} % Change X to a consecutive number; for referencing this chapter elsewhere, use \ref{ChapterX}

通过本次实验,学习到了数据库的基本结构以及SQL语句的执行方式。通过C++代码完成了一个简单的数据库,使我们更加深入地理解了数据库的功能与实现。

在数据库的框架上存在一些问题,助教提供的页式文件系统在windows上存在问题,每一页都会多出一个奇怪的符号,最终只能讲所有代码在linux下实现。

本次实验要求的功能使用stanford的框架比较难实现,包括需要在where域支持and和or操作等等,因此最后几个部分的接口均由自己设计。但是由于考虑不周,最后实现起来select的函数和之前几个部分所采取的方法完全不同,导致查询和输出这一部分代码差不多写了两遍,在之后项目中应该深思熟虑后确定好一个好的框架后再动手写代码。

我们建议调整一下分值,基础模块的分值提高,并把索引模块加入到基础模块中,原因在于基础模块的代码量远大于附加功能的代码量,在和多个同学的讨论中同学们一致认为,附加功能难度并不大,而基础功能则比较繁琐,难度更高。